\usepackage[latin1]{inputenc} 
%\usepackage[dutch]{babel} does not work with IEEE translations
%\usepackage{a4wide,times}


\usepackage{amsmath} %AMS for formulas
\usepackage{amsfonts}
\usepackage{amssymb}

\usepackage{bm} %bold mathmode

\usepackage{graphicx} %for images: \includegraphics{}
\usepackage{wrapfig}  %for images wrapped in text
\usepackage{epstopdf} %for inserting eps images (from matlab)

%\usepackage[headings,cm]{fullpage} %grotere paginasize
%\usepackage[margin=1in]{geometry} %custom geometry

%\usepackage[left=2.5cm,right=2.5cm,top=3.5cm,bottom=2.5cm]{geometry}

\usepackage{float}		%%%%% voor [H]

\usepackage{siunitx}
%\sisetup{load-configurations = abbreviations}

\usepackage{multicol} %multicolom dingen

\usepackage{cases} %voor { met meerdere formules en labels

\usepackage{eso-pic} %Voor TU-bies en blokken
\usepackage{nextpage}			% Advanced nextpage commands
\usepackage{picture}

\usepackage{standalone} %voor gefixte \input 
\usepackage{booktabs} %voor mooie tabellen met \toprule \midrule en \bottomrule
\usepackage{verbatim} %voor input text
\let\labelindent\relax %Fix IEEE legacy problems
\usepackage{enumitem}   % voor mooie enumerate 
%\usepackage{dcolumn} 	% to center table columns on specific characters




%\usepackage[super]{cite}	%om je cite style aan te passen (\renewcommand{\citeleft}{}\renewcommand{\citeright}{}





%\usepackage{multirow} %voor meerdere rijen in tabellen
\usepackage{fix-cm} %%%%%
\newcommand{\matlab}{{\textsc{matlab }}} %MATLAB moet geschreven worden in smallcaps. Dit defineerd het commando \matlab dat dit doet


\usepackage{calc}

%\usepackage{caption}
%\usepackage{subcaption}

\usepackage[T1,hyphens]{url}
\usepackage{hyperref}
\usepackage[nameinlink]{cleveref} %\Cref for \autoref with a capital

%\autoref{} dingen
%\AtBeginDocument{%otherwise it won't work in preamble
%\renewcommand{\tableautorefname}{table}
%\renewcommand{\figureautorefname}{figure}
%\renewcommand{\chapterautorefname}{chapter}
%\renewcommand{\sectionautorefname}{section}
%\renewcommand{\subsectionautorefname}{section}
%\renewcommand{\subsubsectionautorefname}{section}
%\renewcommand{\equationautorefname}{equation}
%\renewcommand\lstlistingname{Code}
%}


%dit is zodat hyperlinks niet blauw zijn.
\hypersetup{
    colorlinks,
    citecolor=black,
    filecolor=black,
    linkcolor=black,
    urlcolor=black
}

%tikz:
\usepackage{pgf}
\usepackage{pgfplots}
\usepackage{pgfgantt}
\pgfplotsset{compat=1.7}
\usepackage{tikz}
\usetikzlibrary{arrows.meta,automata,shapes,calc,positioning,decorations.markings,matrix}
\newcommand{\setpathasarrows}{\tikzstyle{every path}=[auto,line width=1.5pt,line cap=round,line join=round]}


%TODO fix black diamond in decision center
\tikzset{%for flow charts casper
    treenode/.style = {shape=rectangle,
                        draw=black,anchor=center,
                        fill=tucyaan,align=center,
                        line width=1pt,
                        text centered
                        %inner sep=1ex},
                        },
    start/.style = {treenode, circle, inner sep=0pt,minimum width=1.5cm},
    decision/.style = {treenode, diamond,text badly centered,inner sep=0pt,minimum width = 1.8cm,text width=4cm,aspect=4},
    action/.style   = {treenode,        align=center,minimum width = 1.8cm,text width=3cm},
    action_wide/.style   = {treenode,        align=center,minimum width = 1.8cm,text width=4cm},
    statedashed/.style={state,circle,dashed,draw=black,fill=none},
    goto/.style = {treenode,isosceles triangle,align=center,text width=3cm},
    hidden/.style = {treenode,coordinate,fill=none,anchor=center,minimum width = 0.5cm},
    line_custom/.style = {-,thick,auto,draw=black,line width=1.5pt},
    >={Latex[width=3mm,length=3mm]}
}


%\usepackage{subfigure}

\usepackage{xifthen}

\usepackage{comment}




%define TU colors:
\usepackage{xcolor}


\definecolor{tublauw}{RGB}{000,166,214}%for backwards compatibility


%TU corporate colors
\definecolor{tucyaan}{RGB}{000,166,214}
\definecolor{tuzwart}{RGB}{000,000,000}
\definecolor{tuwit}{RGB}{255,255,255}
%TU basiskleuren
\definecolor{tuzeegroen}{RGB}{102,188,170}
\definecolor{tupaars}{RGB}{15,17,80}
\definecolor{tugroen}{RGB}{0,122,133}
\definecolor{tuturquoise}{RGB}{0,147,171}
\definecolor{tudonkerblauw}{RGB}{0,43,96}
\definecolor{tuhemelblauw}{RGB}{119,192,215}

%TU accentkleuren
\definecolor{tulavendel}{RGB}{123,160,201}
\definecolor{tufuchsia}{RGB}{161,0,88}
\definecolor{tuoranje}{RGB}{236,127,44}
\definecolor{tuhelgroen}{RGB}{173,198,16}
\definecolor{tuwarmpaars}{RGB}{131,38,124}
\definecolor{tugeel}{RGB}{247,235,144}


%colors for syntax highlighting:
\colorlet{keyword}{blue!100!black!80}
\colorlet{comment}{green!50!black!100}




\definecolor{dkgreen}{rgb}{0,0.6,0}
\definecolor{gray}{rgb}{0.5,0.5,0.5}

%Listing setup:
\usepackage{listings}

\begin{comment}%some styles for VHDL, propably not the best way to do it.
\lstset{numbers		= left,			%
		numberstyle	= \tiny,		%
		numbersep	= 5pt,			%
		language	= VHDL,			%
		breaklines	= true,			%
		showspaces	= false,		%
		showstringspaces= false}

\lstdefinestyle{vhdl}{
  language     = VHDL,
  basicstyle   = \ttfamily,
  keywordstyle = \color{keyword}\bfseries,
  commentstyle = \color{dkgreen}
}
\end{comment}


%basicstyle=\ttfamily\footnotesize
\lstset{language=Matlab,
keywords={break,case,catch,continue,else,elseif,end,for,function,global,if,otherwise,persistent,return,switch,try,while},
   basicstyle=\ttfamily\footnotesize,
   keywordstyle=\color{blue},
   commentstyle=\color{dkgreen},
   stringstyle=\color{red},
   numbers=left,
   numberstyle=\tiny\color{gray},
   stepnumber=1,
   numbersep=10pt,
   backgroundcolor=\color{white},
   tabsize=4,
   showspaces=false,
   showstringspaces=false}



%Voor laagstreepje in Section headers: 
%dit geeft 1 warning
%\usepackage[T1]{fontenc}
%\catcode`_=12
%\begingroup\lccode`~=`_\lowercase{\endgroup\let~\sb}
%\mathcode`_="8000


%overline above text outside math
%\makeatletter
%\newcommand*{\textoverline}[1]{$\overline{\hbox{#1}}\m@th$}
%\makeatother




%Custom references:
\newcommand{\bigref}[1]			%result:"subsection 1.1: ‘Introduction’"
{\autoref{#1}: `\nameref{#1}'}

\newcommand{\bigrefs}[1]		%result:"1.1: ‘Introduction’"
{\ref{#1}: `\nameref{#1}'}

\newcommand{\secref}[1]		%result:"1.1: ‘Introduction’"
{\ref{#1}: `\nameref{#1}'}

%non-breaking math mode
\newcommand{\nbmm}[1]{%
  {}$% get out of math
  \kern-2\mathsurround % in case it's non zero
  $% reenter math
  \binoppenalty10000 \relpenalty10000 #1% typeset the subformula
  {}$% get out of math
  \kern-2\mathsurround % in case it's non zero
  $% reenter math for the rest of the formula
}

%\setlength\parindent{0pt} %no par indent
%\usepackage[parfill]{parskip}


%\renewcommand{\thefootnote}{\roman{footnote}}

