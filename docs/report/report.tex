%*******************************************************************************%
%===================================Information=================================%
%			Author: 	Casper van Wezel										%
%			Template: 	Based on the IEEE template								%
%																				%
%						IoT														%
%						2017-11-16												%
%*******************************************************************************%

\documentclass[a4paper,journal]{DDREAM}

% correct bad hyphenation here
\hyphenation{op-tical net-works semi-conduc-tor}
%\usepackage[dutch]{babel} 

\usepackage{import}
\inputfrom{./}{macros}
\newlength\figureheight
\newlength\figurewidth
\usepackage[percent]{overpic}
%\usepackage{siunitx}
%\usepackage{hyperref}
%\usepackage{tikz}
%\usetikzlibrary{arrows,automata,shapes,calc,positioning,decorations.markings}

%**********************************************************************************
%=============================Begin Document=====================================%
%**********************************************************************************

\begin{document}
%
% paper title
% can use linebreaks \\ within to get better formatting as desired
\title{\vspace*{0.0cm} Solar Irradiance Prediction \\ using IoT-enabled Solar Modules}
%
%
% author names and IEEE memberships
% note positions of commas and nonbreaking spaces ( ~ ) LaTeX will not break
% a structure at a ~ so this keeps an author's name from being broken across
% two lines.
\author{\vspace*{0.0cm}Anirudh~Bisht~,
Nikolas~Skartsilas,
Casper~van~Wezel,~\IEEEmembership{Delft University of Technology}% <-this % stops a space
\thanks{\footnotesize{This group of Embedded Systems Master students from Delft University of Technology is formed to work togethor on this project. This report is written as part of the course `Internet of Things Seminar' (IN4398 2016/2017).}}%
}

%--------------------------------------------------------------------------------
% The paper headers
\markboth{\strangesize{IoT IN4398 2016/2017, TU Delft}}%
{C. van Wezel}%
% The only time the second header will appear is for the odd numbered pages
% after the title page when using the twoside option.
%--------------------------------------------------------------------------------
% make the title area
\maketitle


%\boldmath
\begin{abstract}
Some random Abstract
\end{abstract}
% IEEEtran.cls defaults to using nonbold math in the Abstract.
% This preserves the distinction between vectors and scalars. However,
% if the journal you are submitting to favors bold math in the abstract,
% then you can use LaTeX's standard command \boldmath at the very start
% of the abstract to achieve this. Many IEEE journals frown on math
% in the abstract anyway.

\begin{IEEEkeywords}
Internet of Things, IoT, Solar, Solar Modules, Irradiance Prediction, Solar Power Prediction, Microgrid Prediction
\end{IEEEkeywords}

\section{Introduction}\label{sec:introduction}
\IEEEPARstart{T}{here} are a ...
\cite{SoCeBa}

\section{Problem Definition}\label{sec:problem_definition}
\IEEEPARstart{O}{f} course it is 


\section{Important Factors or Something Else}\label{sec:factors}
\IEEEPARstart{D}{ifferent} aspects of the destination country have to be looked during the first orientation phase of offshoring. Below are 5 sections which each highlight a different topic, within these topics multiple important factors will be elaborated on.

\subsection{Just a random SubSection}\label{sec:factors-partoffactors}


\section{Conclusion}\label{sec:conclusion}
\IEEEPARstart{S}{ummarizing} one can for sure conclude that 

%\appendices
%\section{Proof of the First Zonklar Equation}
%Appendix one text goes here.

% you can choose not to have a title for an appendix
% if you want by leaving the argument blank
%\section{}
%Appendix two text goes here.


%\section*{Acknowledgment}

% Can use something like this to put references on a page
% by themselves when using endfloat and the captionsoff option.
\ifCLASSOPTIONcaptionsoff
  \newpage
\fi

% trigger a \newpage just before the given reference
% number - used to balance the columns on the last page
% adjust value as needed - may need to be readjusted if
% the document is modified later
%\IEEEtriggeratref{8}
% The "triggered" command can be changed if desired:
%\IEEEtriggercmd{\enlargethispage{-5in}}

% references section

% can use a bibliography generated by BibTeX as a .bbl file
% BibTeX documentation can be easily obtained at:
% http://www.ctan.org/tex-archive/biblio/bibtex/contrib/doc/
% The IEEEtran BibTeX style support page is at:
% http://www.michaelshell.org/tex/ieeetran/bibtex/
%\bibliographystyle{IEEEtran}
% argument is your BibTeX string definitions and bibliography database(s)
%\bibliography{IEEEabrv,../bib/paper}
%
% <OR> manually copy in the resultant .bbl file
% set second argument of \begin to the number of references
% (used to reserve space for the reference number labels box)

\begin{thebibliography}{1}

\bibitem{SoCeBa}
E. de Haan; R. Plaisant van der Wal, C. van Wezel, R. Zwetsloot, \emph{Solar Cell Balancing}. BSc. Electical Engineering Thesis, Delft University of Technology, 2016.

\end{thebibliography}

\end{document}
