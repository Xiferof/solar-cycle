%*******************************************************************************%
%===================================Information=================================%
%			Author: 	Casper van Wezel										%
%			Template: 	Based on the IEEE template								%
%																				%
%						IoT														%
%						2017-11-16												%
%*******************************************************************************%

\documentclass[a4paper,journal]{DDREAM}

% correct bad hyphenation here
\hyphenation{op-tical net-works semi-conduc-tor}
%\usepackage[dutch]{babel} 

\usepackage{import}
\inputfrom{./}{macros}
\newlength\figureheight
\newlength\figurewidth
\usepackage[percent]{overpic}
%\usepackage{siunitx}
%\usepackage{hyperref}
%\usepackage{tikz}
%\usetikzlibrary{arrows,automata,shapes,calc,positioning,decorations.markings}

%**********************************************************************************
%=============================Begin Document=====================================%
%**********************************************************************************

\begin{document}
%
% paper title
% can use linebreaks \\ within to get better formatting as desired
\title{\vspace*{0.0cm} Solar Irradiance Prediction \\ using IoT-enabled Solar Modules}
%
%
% author names and IEEE memberships
% note positions of commas and nonbreaking spaces ( ~ ) LaTeX will not break
% a structure at a ~ so this keeps an author's name from being broken across
% two lines.
\author{\vspace*{0.0cm}Anirudh~Bisht~,
Nikolas~Skartsilas,
Casper~van~Wezel,~\IEEEmembership{Delft University of Technology}% <-this % stops a space
\thanks{\footnotesize{This group of Embedded Systems Master students from Delft University of Technology is formed to work togethor on this project. This report is written as part of the course `Internet of Things Seminar' (IN4398 2016/2017).}}%
}

%--------------------------------------------------------------------------------
% The paper headers
\markboth{\strangesize{IoT IN4398 2016/2017, TU Delft}}%
{C. van Wezel}%
% The only time the second header will appear is for the odd numbered pages
% after the title page when using the twoside option.
%--------------------------------------------------------------------------------
% make the title area
\maketitle


%\boldmath
\begin{abstract}
Some random Abstract
\end{abstract}
% IEEEtran.cls defaults to using nonbold math in the Abstract.
% This preserves the distinction between vectors and scalars. However,
% if the journal you are submitting to favors bold math in the abstract,
% then you can use LaTeX's standard command \boldmath at the very start
% of the abstract to achieve this. Many IEEE journals frown on math
% in the abstract anyway.

\begin{IEEEkeywords}
Internet of Things, IoT, Solar, Solar Modules, Irradiance Prediction, Solar Power Prediction, Microgrid Prediction
\end{IEEEkeywords}

\section{Introduction}\label{sec:introduction}
\IEEEPARstart{T}{here} are a more and more photovoltaic systems deployed world wide every day.
This is all done with the mindset to harvest as much solar energy as possible in order to provide a sustainable electricity source for our increasing needs.
Experts in this area see it as a real posibility that in the future our energy grid will be based on a lot of smaller electricity sources like these home PV systems, the so called micro-grid.
One of the main drawbacks of these micro-grid is controllability, because of the sheer number of contributers.
Currently there is a lot of rotating mass connected to the energy grid (because of all the heavy metal turbines), tinny fluctuations in the power flow of the grid are smoothened by this energy buffer.
Bigger fluctations are controlled by the network operator by increasing or decreasing the power generator at the different sites.
Keeping this in mind, a big disadvantage can be seen when looking back at the home PV systems.
To start with, there will be almost no intertia in a micro-grid system and on top of that the power generation is heavily depedent on the current irradiance by the sun and cloud movement.
So this method of generating energy is way less controllable and also really unpredictable.
This proposes some challenges in the future development of the power grid.

Anonther major drawback of current photovoltaic systems is the great impact of partial shading of a panel on the whole system.
Because all the cells are connected in series, a smaller generated current in one cell decreases the current through all cells.
A lot of research is done in this area to solve this not so well known problem, all possibilties are discussed in \cite{SoCeBa}.
The authors of this thesis also propose and design an implementation to solve this problem.
In general, most implementations use a microcontroller which optimizes the power output in someway.
With current trends of all devices being more and more connected to the internet (the Internet of Things era), this might also provide the posibilities to provide a solution for the problem of the predictability of micro-grid PV systems.

In the future, all the PV modules will be 'smart' (i.e. have a microcontroller in them), this can be used to collect huge ammounts of data of the current irradiance.
Using all this data and the geographical locations, the resolution of cloud movement predictions can be enhanced greatly.
These more accurate cloud predictions or models can be fed back into the system in order to make a short term prediction of the power generation of these small PV systems.

\section{Problem Definition}\label{sec:problem_definition}
\IEEEPARstart{O}{f} course it is 


\section{Implementation}\label{sec:implementation}
\IEEEPARstart{D}{ifferent} 

\subsection{Just a random SubSection}\label{sec:implementation-algorithm}


\section{Experimental Setup}\label{sec:experiment}
\IEEEPARstart{T}{he} goal of this experiment is to provide a prototype showing that this method could work in the future.
The main goal of this Experimental Setup thus is to just show the working version of the algorithm.

\subsection{Hardware}\label{sec:implementation-hardware}
Because the main goal of this prototype is to show that the algorithm works, a lot of simplifications can be made on the hardware level.
For example expensive PV cells can be replaced by cheap Light Dependant Resistors (LDRs).
A bunch of cheap Arduino Nanos can be used to represent multiple houses in the grid

\subsection{Algorithm}\label{sec:implementation-algorithm}




\section{Conclusion}\label{sec:conclusion}
\IEEEPARstart{S}{ummarizing} one can for sure conclude that 

%\appendices
%\section{Proof of the First Zonklar Equation}
%Appendix one text goes here.

% you can choose not to have a title for an appendix
% if you want by leaving the argument blank
%\section{}
%Appendix two text goes here.


%\section*{Acknowledgment}

% Can use something like this to put references on a page
% by themselves when using endfloat and the captionsoff option.
\ifCLASSOPTIONcaptionsoff
  \newpage
\fi

% trigger a \newpage just before the given reference
% number - used to balance the columns on the last page
% adjust value as needed - may need to be readjusted if
% the document is modified later
%\IEEEtriggeratref{8}
% The "triggered" command can be changed if desired:
%\IEEEtriggercmd{\enlargethispage{-5in}}

% references section

% can use a bibliography generated by BibTeX as a .bbl file
% BibTeX documentation can be easily obtained at:
% http://www.ctan.org/tex-archive/biblio/bibtex/contrib/doc/
% The IEEEtran BibTeX style support page is at:
% http://www.michaelshell.org/tex/ieeetran/bibtex/
%\bibliographystyle{IEEEtran}
% argument is your BibTeX string definitions and bibliography database(s)
%\bibliography{IEEEabrv,../bib/paper}
%
% <OR> manually copy in the resultant .bbl file
% set second argument of \begin to the number of references
% (used to reserve space for the reference number labels box)

\begin{thebibliography}{1}

\bibitem{SoCeBa}
E. de Haan; R. Plaisant van der Wal, C. van Wezel, R. Zwetsloot, \emph{Solar Cell Balancing}. BSc. Electical Engineering Thesis, Delft University of Technology, 2016.

\end{thebibliography}

\end{document}
